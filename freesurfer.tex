\documentclass[12pt]{article}

\usepackage{amsmath, amssymb}
\usepackage{geometry}
\usepackage{hyperref}

\geometry{margin=1in}

\title{Core Formulas of Freesurfer}
\author{Jules}
\date{\today}

\begin{document}

\maketitle

\tableofcontents

\section{Surface Reconstruction: Energy Minimization}

The reconstruction of the cortical surface is framed as an energy minimization problem. A deformable model (a mesh) is iteratively adjusted to minimize a total energy function, $E_{total}$, which balances surface smoothness with fidelity to the MRI data.

\begin{equation}
E_{total}(S) = E_{internal}(S) + E_{external}(S)
\end{equation}

\subsection{Internal Energy ($E_{internal}$)}
This term enforces the \textbf{smoothness} of the surface, $S$. It's often modeled like a thin plate, penalizing stretching and bending. A simplified form is:
\begin{equation}
E_{internal}(S) = \int_{S} \alpha(s) \left\| \frac{\partial S}{\partial s} \right\|^2 + \beta(s) \left\| \frac{\partial^2 S}{\partial s^2} \right\|^2 ds
\end{equation}
\begin{itemize}
    \item The first term, weighted by $\alpha$, represents the \textbf{tension} (resistance to stretching).
    \item The second term, weighted by $\beta$, represents the \textbf{rigidity} (resistance to bending).
\end{itemize}

\subsection{External Energy ($E_{external}$)}
This term pulls the surface toward the desired anatomical boundary (e.g., the white matter/gray matter interface). It's based on the image intensity, $I$, and is strongest where the image gradient, $\nabla I$, is high.
\begin{equation}
E_{external}(S) = - \int_{S} \left| \nabla (G_{\sigma} * I(s)) \right|^2 ds
\end{equation}
\begin{itemize}
    \item $I(s)$ is the image intensity at a point $s$ on the surface.
    \item $G_{\sigma} * I$ represents the image convolved with a Gaussian kernel, which smooths the data.
    \item The negative sign ensures the energy is minimized when the surface is at a location of a strong intensity gradient.
\end{itemize}

The surface is evolved using \textbf{gradient descent}, where the position of the surface at time $t+1$ is updated based on the functional derivative of the energy:
\begin{equation}
S_{t+1} = S_t - \gamma \frac{\delta E_{total}}{\delta S}
\end{equation}
where $\gamma$ is the learning rate.

\section{Subcortical Segmentation: Bayesian Classification}

Automated segmentation of structures like the hippocampus and thalamus relies on a probabilistic atlas and Bayesian classification. The goal is to find the most probable anatomical label, $L$, for each voxel, given its intensity, $I$, and its location within a coordinate system.

The core formula is \textbf{Bayes' theorem}:
\begin{equation}
P(L_i | I_i) = \frac{P(I_i | L_i) \cdot P(L_i)}{\sum_{L} P(I_i | L) \cdot P(L)} \propto P(I_i | L_i) \cdot P(L_i)
\end{equation}
\begin{itemize}
    \item $P(L_i | I_i)$ is the \textbf{posterior probability}: the probability of label $L_i$ given the observed intensity $I_i$.
    \item $P(I_i | L_i)$ is the \textbf{likelihood}: the probability of observing intensity $I_i$ if the true label is $L_i$. This is learned from a manually labeled training set, often modeled as a Gaussian mixture.
    \item $P(L_i)$ is the \textbf{spatial prior}: the probability of label $L_i$ occurring at this specific location, irrespective of intensity. This is derived from the probabilistic atlas. Freesurfer enhances this prior using a \textbf{Markov Random Field (MRF)}, which adds a term that encourages neighboring voxels to have the same label, ensuring spatial coherence.
\end{itemize}

\section{Cortical Thickness Calculation}

Cortical thickness is not a single formula but an algorithmic measurement based on the shortest distance between the white matter surface ($S_w$) and the pial surface ($S_p$).

For a vertex $v_w$ on the white surface, the distance to the pial surface is found. For the corresponding vertex $v_p$ on the pial surface, the distance to the white surface is found. The thickness, $T$, at that location is the average of these two measurements.

\begin{equation}
T(v) = \frac{1}{2} \left( \min_{p \in S_p} \|v_w - p\| + \min_{w \in S_w} \|v_p - w\| \right)
\end{equation}
\begin{itemize}
    \item $\min_{p \in S_p} \|v_w - p\|$ calculates the shortest Euclidean distance from the white matter vertex $v_w$ to any point on the pial surface $S_p$.
    \item $\min_{w \in S_w} \|v_p - w\|$ calculates the shortest Euclidean distance from the pial vertex $v_p$ to any point on the white matter surface $S_w$.
\end{itemize}

\section{Statistical Analysis: The General Linear Model (GLM)}

To test for group differences or correlations with variables like age, Freesurfer applies the GLM at every vertex on the cortical surface.

The model is expressed in matrix form as:
\begin{equation}
Y = X\beta + \epsilon
\end{equation}
\begin{itemize}
    \item $Y$ is the \textbf{data vector} (e.g., cortical thickness values at one vertex for all subjects).
    \item $X$ is the \textbf{design matrix}, where each row represents a subject and each column represents a predictor variable (e.g., group membership, age, sex).
    \item $\beta$ is the \textbf{parameter vector} of weights that are estimated. These values represent the magnitude of the effect of each predictor.
    \item $\epsilon$ is the \textbf{error vector} (residuals).
\end{itemize}

The parameters $\beta$ are estimated using the \textbf{ordinary least squares} solution:
\begin{equation}
\hat{\beta} = (X^T X)^{-1} X^T Y
\end{equation}

To test a specific hypothesis (e.g., is there a difference between two groups?), a \textbf{contrast vector} $c$ is used to define the hypothesis, and a \textbf{t-statistic} is calculated:
\begin{equation}
t = \frac{c^T \hat{\beta}}{\sqrt{\hat{\sigma}^2 c^T (X^T X)^{-1} c}}
\end{equation}
\begin{itemize}
    \item $\hat{\sigma}^2$ is the estimated variance of the residuals.
\end{itemize}
This calculation is repeated for all vertices, producing a statistical map on the brain surface.

\end{document}
