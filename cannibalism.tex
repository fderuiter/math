\documentclass[12pt]{article}

\usepackage{amsmath, amssymb}
\usepackage{geometry}
\usepackage{hyperref}

\geometry{margin=1in}

\title{Mathematical Models of Cannibalism}
\author{Jules}
\date{\today}

\begin{document}

\maketitle

\section{The McKendrick-von Foerster Equation}

This equation is a cornerstone of age-structured population modeling and is adapted in the paper to model cannibalism.

\textbf{Formula:}
\begin{equation}
\frac{\partial n(t, a)}{\partial t} + \frac{\partial n(t, a)}{\partial a} = -\mu(t, a) n(t, a)
\end{equation}
\textbf{Boundary Condition:}
\begin{equation}
n(t, 0) = b(t)
\end{equation}

\textbf{Description of Variables:}
\begin{itemize}
    \item \(n(t, a)\): The number of individuals of age \(a\) at time \(t\).
    \item \(\mu(t, a)\): The per capita death rate of individuals of age \(a\) at time \(t\).
    \item \(b(t)\): The birth rate of the population at time \(t\).
\end{itemize}

\section{Death Rate Equation}

This equation, from the model by Diekmann et al., provides a specific formulation for the death rate, incorporating both natural mortality and cannibalism.

\textbf{Formula:}
\begin{equation}
\mu(t, a) = \nu(a) + C(a)k(t)\Phi(c(t))
\end{equation}

\textbf{Description of Variables:}
\begin{itemize}
    \item \(\mu(t, a)\): The per capita death rate.
    \item \(\nu(a)\): The natural death rate for an individual of age \(a\).
    \item \(C(a)\): The attack rate of cannibals on individuals of age \(a\).
    \item \(k(t)\): The total number of cannibals at time \(t\).
    \item \(\Phi(c(t))\): A density-dependent correction factor, which accounts for how the cannibalism rate might change with the density of the cannibal population \(c(t)\).
\end{itemize}

\section{Juvenile and Adult Dynamics Equations}

These equations, from the model by van den Bosch et al., separate the population into juvenile and adult stages to model cannibalism where adults prey on juveniles.

\textbf{Juvenile Dynamics Equation:}
\begin{equation}
\frac{\partial n(t, a)}{\partial t} + \frac{\partial n(t, a)}{\partial a} = -[I_t + C(a)A(t)]n(t, a)
\end{equation}

\textbf{Adult Dynamics Equation:}
\begin{equation}
\frac{dA}{dt} = n(t, \alpha) - f(I(t))A(t)
\end{equation}

\textbf{Description of Variables:}
\begin{itemize}
    \item \(n(t, a)\): The number of juvenile individuals of age \(a\) at time \(t\).
    \item \(I_t\): The per capita natural death rate of juveniles.
    \item \(C(a)\): The attack rate of an adult on a juvenile of age \(a\).
    \item \(A(t)\): The total number of adults at time \(t\).
    \item \(\alpha\): The age at which juveniles mature into adults.
    \item \(f(I(t))\): The per capita death rate of adults, which is a function of their energy intake \(I(t)\).
\end{itemize}

\section{Two-Dimensional ODE Model for Cannibalism}

This model simplifies the population into two groups: "normal" individuals and "cannibalistic" individuals, using a system of ordinary differential equations.

\textbf{Formulas:}
\begin{align}
\frac{dN}{dt} &= \beta_N(N, C)N + \beta_C(N, C)C - K(N)N - \phi(N, C) - \mu_N(N, C)N \\
\frac{dC}{dt} &= K(N)N - \mu_C(N, C)C
\end{align}

\textbf{Description of Variables:}
\begin{itemize}
    \item \(N\): The number of normal individuals.
    \item \(C\): The number of cannibalistic individuals.
    \item \(\beta_N(N, C)\) and \(\beta_C(N, C)\): The birth rates of normal and cannibalistic individuals, respectively, which can depend on the numbers of both types of individuals.
    \item \(K(N)\): The rate at which normal individuals become cannibals, which can depend on the number of normal individuals.
    \item \(\phi(N, C)\): A term representing the loss of normal individuals due to cannibalism.
    \item \(\mu_N(N, C)\) and \(\mu_C(N, C)\): The death rates of normal and cannibalistic individuals, respectively.
\end{itemize}

\end{document}
