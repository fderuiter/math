\documentclass[12pt]{article}

\usepackage{amsmath, amssymb}
\usepackage{geometry}
\usepackage{graphicx}
\usepackage{hyperref}
\usepackage{listings}
\usepackage{xcolor}

\geometry{margin=1in}

\lstset{
    basicstyle=\ttfamily\small,
    keywordstyle=\color{blue},
    commentstyle=\color{gray},
    stringstyle=\color{red},
    showstringspaces=false,
    numbers=left,
    numberstyle=\tiny\color{gray},
    breaklines=true,
    frame=single,
    language=Python
}

\begin{document}

\title{A Satirical Analysis of Maternal Favoritism}
\author{Your Name}
\date{\today}

\maketitle

\begin{abstract}
This essay presents a humorous and satirical mathematical model to calculate the favoritism score of a child from their mother. By blending complex mathematical constructs with whimsical assumptions, we derive a comprehensive formula and implement it using Python. The model considers factors such as proximity, emotional support, gift-giving, compliments, birth order, personality traits, and random fluctuations to capture the essence of maternal favoritism.
\end{abstract}

\tableofcontents

\section{Introduction}

Understanding the dynamics of maternal favoritism has been a topic of both scientific inquiry and familial debate. In this satirical essay, we explore an exaggerated mathematical approach to quantify favoritism using an array of integrals, matrices, and stochastic processes. While the model is humorous in nature, it serves to highlight the complexity and subjectivity involved in familial relationships.

\section{The Favoritism Formula}

The favoritism score \( F \) is determined by a complex function incorporating various factors:

\subsection{Proximity Integral}
The cumulative effect of being close to mom over a given time period \( T \):

\begin{equation}
\int_0^T \frac{1}{x(t)} \, dt
\end{equation}

\subsection{Emotional Support Integral}
The emotional support you provide, calculated as a double integral over space and time:

\begin{equation}
\iint_S e(x, t) \, dx \, dt
\end{equation}

\subsection{Gift-Giving Matrix Determinant}
A matrix transformation representing the value of emotional and practical gifts:

\begin{equation}
\det(G)
\end{equation}

\subsection{Compliments Score}
The dot product of the number of compliments and their weighted impact based on mom’s values:

\begin{equation}
\vec{C} \cdot \vec{w}
\end{equation}

\subsection{Frequency of Contact}
A logarithmic term that accounts for diminishing returns on how often you contact mom:

\begin{equation}
\log(1 + f(t))
\end{equation}

\subsection{Personality Score \( P_{\text{total}} \)}
A weighted sum of key personality traits—intelligence, emotional sensitivity, wealth, and talents:

\begin{equation}
P_{\text{total}} = w_I \cdot I + w_{E_s} \cdot E_s + w_W \cdot W + w_T \cdot T
\end{equation}

\subsection{Birth Order \( B \)}
A refinement of the birth order effect, including factors like intelligence or emotional connection:

\begin{equation}
B = B_1, B_2, B_3, B_4
\end{equation}

\subsection{Major Life Events \( M \)}
Adds points for significant life events:

\begin{equation}
M = \sum m_i
\end{equation}

\subsection{Health Crisis Factor \( H \)}
Multiplies favoritism if you helped during a crisis:

\begin{equation}
H = \begin{cases}
1.5 & \text{if you were there} \\
1 & \text{otherwise}
\end{cases}
\end{equation}

\subsection{Social Media Activity \( S \)}
Adds a multiplier for engaging with mom’s social media posts:

\begin{equation}
S = \begin{cases}
1.3 & \text{if active} \\
1 & \text{otherwise}
\end{cases}
\end{equation}

\subsection{Favorability Decay \( D \)}
Decays over time since your last contact with mom:

\begin{equation}
D = e^{-k \cdot t_{\text{since\_last\_contact}}}
\end{equation}

\subsection{Sibling Proximity Sum}
Accounts for the proximity of your siblings to mom, reducing your score if they’re closer:

\begin{equation}
\sum_{i=1}^{n} \frac{1}{S_i(t)}
\end{equation}

\subsection{Randomness \( R \)}
Adds random fluctuations based on mom's unpredictable mood:

\begin{equation}
R = \text{Random factor from stochastic processes}
\end{equation}

\section{Final Favoritism Formula}

Combining all the components, we arrive at the complete formula:

\begin{equation}
F = \frac{\left( \displaystyle \int_0^T \frac{1}{x(t)} \, dt \right) \cdot \left( \displaystyle \iint_S e(x, t) \, dx \, dt \right) \cdot \det(G) \cdot \left( \vec{C} \cdot \vec{w} \right) \cdot \log(1 + f(t)) \cdot P_{\text{total}} \cdot B \cdot M \cdot H \cdot S \cdot D \cdot R}{\displaystyle \int_0^T \sum_{i=1}^{n} \frac{1}{S_i(t)} \, dt}
\end{equation}

\section{Python Implementation of the Favoritism Formula}

Here’s how you could implement this satirical favoritism calculation in Python:

\begin{lstlisting}[language=Python]
import numpy as np
from scipy.integrate import quad, dblquad
import random

# Constants and parameters
T = 365
g_emotional = 5
g_practical = 2
f_initial = 7
birth_order_weight = 1.2
major_life_events = 3
helped_during_crisis = True
H = 1.5 if helped_during_crisis else 1
active_on_social_media = True
S = 1.3 if active_on_social_media else 1
decay_constant = 0.05
time_since_last_contact = 7
D = np.exp(-decay_constant * time_since_last_contact)
intelligence = 7
emotional_sensitivity = 6
wealth = 9
talent = 8
w_I = 1.2
w_Es = 1.5
w_W = 1.1
w_T = 1.3
sibling_distances = [100, 50, 10]
compliments = {'cooking': 10, 'appearance': 5, 'intelligence': 8}
compliment_weights = {'cooking': 1, 'appearance': 0.5, 'intelligence': 0.75}
R = random.uniform(0.9, 1.1)
x_0 = 20

# Functions for integrals
def proximity(t):
    return 1 / x_0

def emotional_support(x, t):
    return 8

def sibling_proximity(t):
    return sum(1 / distance for distance in sibling_distances)

# Calculations
proximity_integral = quad(proximity, 0, T)[0]
emotional_support_integral = dblquad(emotional_support, 0, T, lambda t: 0, lambda t: 1)[0]
gift_matrix = np.array([[g_emotional, 0], [0, g_practical]])
gift_matrix_determinant = np.linalg.det(gift_matrix)
compliment_values = np.array([compliments[key] for key in compliment_weights])
compliment_weight_values = np.array([compliment_weights[key] for key in compliment_weights])
compliment_score = np.dot(compliment_values, compliment_weight_values)
frequency_term = np.log(1 + f_initial)
personality_score = w_I * intelligence + w_Es * emotional_sensitivity + w_W * wealth + w_T * talent
sibling_proximity_integral = quad(sibling_proximity, 0, T)[0]

numerator = (
    proximity_integral *
    emotional_support_integral *
    gift_matrix_determinant *
    compliment_score *
    frequency_term *
    personality_score *
    birth_order_weight *
    major_life_events *
    H *
    S *
    D *
    R
)

denominator = sibling_proximity_integral

F = numerator / denominator

print(f"Favoritism Score: {F}")
\end{lstlisting}

\section{Explanation of Key Steps}

\subsection{Personality Traits}
Each trait (intelligence, emotional sensitivity, wealth, talents) contributes to the total personality score:

\begin{equation}
P_{\text{total}} = w_I \cdot I + w_{E_s} \cdot E_s + w_W \cdot W + w_T \cdot T
\end{equation}

\subsection{Compliments}
Calculated using the dot product between the number of compliments and the weight each type of compliment carries:

\begin{equation}
\vec{C} \cdot \vec{w} = \sum_{i} C_i \cdot w_i
\end{equation}

\subsection{Gift Matrix}
Determinant of the gift-giving matrix, which includes both emotional and practical gifts:

\begin{equation}
\det(G) = g_{\text{emotional}} \cdot g_{\text{practical}} - 0
\end{equation}

\subsection{Proximity and Emotional Support Integrals}
Calculate the cumulative effect of being close and providing emotional support over time:

\begin{align}
\text{Proximity Integral} &= \int_0^T \frac{1}{x(t)} \, dt \\
\text{Emotional Support Integral} &= \iint_S e(x, t) \, dx \, dt
\end{align}

\subsection{Favorability Decay}
Ensures favoritism decays over time if you haven’t contacted mom recently:

\begin{equation}
D = e^{-k \cdot t_{\text{since\_last\_contact}}}
\end{equation}

\subsection{Random Factor}
Adds some randomness to simulate mom’s unpredictable mood swings:

\begin{equation}
R = \text{Uniform random number between } 0.9 \text{ and } 1.1
\end{equation}

\section{Conclusion}

This satirical model humorously combines an array of complex mathematical constructs to quantify something as subjective and nuanced as maternal favoritism. While the formula is intentionally exaggerated and not meant to be taken seriously, it underscores the multifaceted nature of family relationships and the myriad factors that can influence perceptions of favoritism.

\section*{References}

\begin{itemize}
    \item Python documentation: \url{https://docs.python.org/3/}
    \item NumPy library: \url{https://numpy.org/}
    \item SciPy library: \url{https://scipy.org/}
\end{itemize}

\end{document}
